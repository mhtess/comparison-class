\documentclass[letterpaper, 12pt]{article}
\usepackage[]{graphicx}
\usepackage[]{color} % use "amsart" instead of "article" for AMSLaTeX format
\usepackage{geometry}                % See geometry.pdf to learn the layout options. There are lots.
\geometry{letterpaper}                     % ... or a4paper or a5paper or ... 
\usepackage[notocbib]{apacite}
%\usepackage{subfigure}
%\usepackage{alltt}
%\usepackage{etex}     
%\usepackage{amsfonts, amsmath, amssymb, latexsym, mathrsfs, fancyhdr,theorem,  pifont, setspace, verbatim,  qtree, lscape, tipa, linguex, hyperref, wasysym, stmaryrd, natbib,soul, minibox, lipsum, setspace, amssymb, color, multirow, multicol, soul,geometry,graphicx, wrapfig,gb4e,booktabs}
%\usepackage[T1]{fontenc}
%\usepackage{times}
%\usepackage{helvet}
%\renewcommand{\familydefault}{\sfdefault}
\geometry{hmargin={1in,1in},vmargin={1in,1in}}
\definecolor{Red}{RGB}{255,0,0}
\graphicspath{{../paper/figs/}}
\usepackage{amssymb,amsmath}
\usepackage{gensymb}
\usepackage{csquotes}
\newcommand{\denote}[1]{\mbox{ $[\![ #1 ]\!]$}}
\usepackage{tikz}
\usepackage{tikz-qtree}
\usepackage{caption}
\usepackage{wrapfig}
\usepackage{pslatex}
\usepackage{apacite}
\usepackage{amsmath}
\usepackage{float} % Roger Levy added this and changed figure/table
                   % placement to [H] for conformity to Word template,
                   % though floating tables and figures to top is
                   % still generally recommended!

%\title{Brief Article}
%\author{Authors}
%\date{ % Activate to display a given date or no date


\begin{document}
\begin{center}
\textbf{Comparison class inference}
\end{center}






A 75 \degree F (24 \degree C) day is warm.
A 60 \degree F (16 \degree C) day is not, unless it's January; it could be warm for January.
\emph{Warm} is relative, and its felicity depends upon what the speaker uses as a basis of comparison---the \emph{comparison class} (e.g., other days of the year or other days in January).
Comparison classes are necessary for understanding adjectives and, in fact, any part of language whose meaning must be pragmatically reconstructed from context, including vague quantifiers \cite<e.g., ``He ate a lot of burgers''; >{Scholler2017} and generic language \cite<e.g., ``Dogs are friendly''; >{Tessler2019psychrev}.
%Deciding on the relevant comparison class is a case study in the larger question of inferring the appropriate aspects of context for interpreting an utterance.
Comparison classes, as with relevant aspects of context more generally, often go unsaid (e.g., in ``It's warm outside'').

How do listeners decide on the appropriate comparison class? 
Any particular referent of discourse can be conceptualized or categorized in multiple ways, giving rise to multiple possible comparison classes: 
A day in January is also a day of the year; if a listener hear \enquote{It's warm}, it could be \emph{warm for the week}, \emph{warm for winter}, or \emph{warm for the year}.\footnote{It could also be \emph{warm for Boston}, \emph{warm for the northeast USA}, \emph{warm for a place with currently six inches of snow on the ground}, among many more possibilities.} 
In this project, we investigate the first aspect of this problem: deciding among multiple plausible, conceptual comparison classes. 
Theoretical work in semantics has focused on how information from the comparison class gets integrated with a compositional semantics and what representations might be preferred \cite{Bale2011, Solt2009}. 
To our knowledge, the question of how listeners reconstruct a comparison class in under-supported contexts (e.g., just hearing \enquote{It's warm outside}) has been addressed neither formally nor empirically.

\begin{wrapfigure}{r}{0.75\textwidth}
%\vspace{-1cm}
  \begin{center}
    \includegraphics[width=0.75\textwidth]{speakerSimulations-1}
  \end{center}
%  \vspace{-0.45cm}
  \caption{\small 
%  Speaker model behavior assuming
%different comparison classes. The underspecified utterance
%(\enquote{warm}) takes on the same meaning of the explicit utterance
%whose comparison class is the same as the implicit class (e.g., if the
%speaker is using the (for winter) comparison class, \enquote{warm} means
%the same thing as \enquote{warm for winter}). When the speaker is
%assuming winter is the comparison class, when the temperture is low, she
%will say \enquote{warm} or \enquote{warm for winter}, but if the
%temperature is relatively high, she will only say \enquote{warm for the
%year}. When the speaker is assuming the year is the comparison class,
%and the temperature is relatively low, she will say \enquote{warm for
%winter}. The listener believes temperature is relatively low (believes
%it is winter), and inverts this model of speaker to conclude that the
%implicit comparison class was winter. The specific class in this example
%(winter) is a Normal distribution centered at -1, and the general class
%(the year) is a Normal centered at 0.
A: Given plausible temperatures in winter (grey distribution), a speaker who assumed the comparison class was \emph{the year} (right facet) would say ``warm for winter'' for typical \emph{warm for winter} temperatures. B: A pragmatic listener who hears on ``it's warm'' during winter reasons that the speaker's comparison class was \emph{winter}.
  }
  \label{fig:incremental}
\end{wrapfigure}


We propose that listeners combine probabilistic, category knowledge with pragmatic reasoning to infer the comparison class implicitly used by the speaker. 
This proposal leads to the following prediction: When it is winter, saying \enquote{it's warm} signals that it is \emph{warm for winter}, while saying \enquote{it's cold} signals \emph{cold for the year}. 
The opposite relationship should hold in summer, where \enquote{it's cold} should signal it's cold \emph{for summer} more so than \enquote{it's warm}. 
This prediction is a result of the \emph{a priori} probability of different temperatures in different seasons: In winter, temperatures are relatively low, and thus it is unlikely to actually be \emph{warm for the year}. 
In addition, regardless of the season and the adjective (e.g., \enquote{warm} or \enquote{cold}), listeners prefer comparison classes that are relatively specific (e.g., relative to \emph{the current season} as opposed to \emph{the whole year}); more specific comparison classes have lower variance, and a vague adjectives like \emph{warm} carries more information when it is interpreted with respect to a lower variance comparison class. 
These predictions fall out of a Rational Speech Act (RSA) model for gradable adjectives \cite{Lassiter2015}, extended to flexibly reason about the implicit comparison class. 
We test these predictions in two experiments by eliciting the comparison class using a forced-choice paraphrase dependent measure (Expt. 1) and a free-production dependent measure using an expanded stimulus set (Expt. 2).




%When processing a conjunctive phrase, listeners may form expectations about the complete utterance even before the sentence is over. 
%For example, when a speaker reaches the word \emph{and} in ``Elephants live in Africa \emph{and}'', she has two syntactically distinct options available to her when completing the sentence: She could continue with a noun phrase (e.g., ``and Asia'') or a verb phrase (e.g., ``and eat bugs'').
%Each of these continuations would imply different inferences about the prevalence of elephants in Africa.\footnote{
%	Of course, it is possible to continue with a verb phrase about a mutually exclusive property such as ``\ldots live in Africa and live in Asia'' as well as continue with a noun phrase about a non-mutually exclusive property (e.g., ``\ldots eat figs and nuts''). Our focus is on the fact that a speaker can continue the conjunction with a mutually exclusive or non-mutually exclusive property, which in the cases we consider, are highly correlated with NP~vs.~VP coordination.
%}
%If listeners parse and interpret utterances incrementally at the level of individual words, then we would expect their 

\begin{figure}[t]
%\vspace{-1cm}
  \begin{center}
    \includegraphics[width=0.8\textwidth]{posteriorPredictiveScatters-1}
  \end{center}
  \vspace{-0.55cm}
  \caption{\small Human endorsement of subordinate comparison class paraphrases (middle; Expt. 1) and adjective sentences (left; Expt. 2) as a function of listener model $L_1$ and speaker model $S_2$ predictions, respectively. The right facet displays a subset of the paraphrase data (Expt. 1) to reveal good quantitative fit even in a small dynamic range. Error bars correspond to 95\% Bayesian credible intervals.}
  \label{fig:results}
%    \vspace{-0.55cm}
\end{figure}


%\begin{wrapfigure}{r}{0.5\textwidth}
%\vspace{-1cm}
%  \begin{center}
%    \includegraphics[width=0.48\textwidth]{expt2_summary}
%  \end{center}
%  \vspace{-1cm}
%  \caption{\small Experiment 1 results.  Participants rate prevalence for mentioned property (\% live in Africa) and ``Property 2'', either the mutually exclusive property (left two bars) or non-mutually exclusive property (right two bars). ``\_\_'' indicates the question appears mid-sentence. Error-bars denote bootstrapped 95\% confidence intervals.}
%  \label{fig:expt1}
%\end{wrapfigure}



%\begin{wrapfigure}{l}{0.5\textwidth}
%\vspace{-1cm}
%  \begin{center}
%    \includegraphics[width=0.48\textwidth]{expt3_summary}
%  \end{center}
%  \caption{\small Experiment 2 results. Participants are interrupted at various stages of the sentence (either after \emph{Africa}, \emph{and}, or \emph{Asia}) to be asked about the prevalence of \emph{living in Africa} and \emph{living in some other place}, or asked at the end of the sentence (right-most bars). When participants are interrupted before the second conjunct (\emph{Asia}), the sentence continues with a non-mutually exclusive property. Error-bars denote bootstrapped 95\% confidence intervals.
%}
%\vspace{-1cm}
%\label{fig:expt2}
%\end{wrapfigure}



%We tested the incremental prediction of the model that when participants only hear ``Elephants live in Africa and'', they will begin to anticipate a mutually-exclusive conjunct, and that this would manifest in their implied prevalence ratings being substantially less than when they only hear ``Elephants live in Africa''.





\vspace{-0.1cm}

\begingroup
\renewcommand{\section}[2]{}
\bibliographystyle{apacite}

\setlength{\bibleftmargin}{.125in}
\setlength{\bibindent}{-\bibleftmargin}

\bibliography{../paper/comparison-class}
\renewcommand\bibname{}
\scriptsize
%Extending an analysis of generics to handle complex-predicates poses unique challenges.
%A sentence of the form ``Ks F and G'' introduces an ambiguity:
%
%\begin{enumerate}
%\tightlist
%\item $\denote{gen}(K) [F \land G]$
%\item $\denote{gen}(K) [F] \land \denote{gen}(K) [G]$
%\end{enumerate}
\endgroup

\end{document}
